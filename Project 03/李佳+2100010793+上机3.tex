\documentclass{article}
\usepackage{graphicx}
\usepackage[UTF8]{ctex}
% \usepackage[showframe]{geometry} %调整页边距showframe显示框架
\usepackage{amsmath}  %数学环境
\usepackage{paralist,bbding,pifont} %罗列环境
\usepackage{lmodern}  %中文环境与amsmath格式冲突
\usepackage{array,graphicx}  %插入表格、图片
\usepackage{float}
\usepackage{appendix}
\usepackage{amssymb}
\usepackage{amsthm}
\usepackage{tocloft}  %目录
\usepackage{listings}
\usepackage{xcolor}
\usepackage{hyperref}
\usepackage{setspace}
\usepackage{algorithm}
\usepackage{algpseudocode}
\usepackage{tikz}
%\documentclass[tikz]{stanalone}	% 
\usepackage{pgfplots}	
\pgfplotsset{compat=newest}
\renewcommand\cftsecdotsep{\cftdotsep}
\renewcommand\cftsecleader{\cftdotfill{\cftsecdotsep}}
\renewcommand {\cftdot}{$ \cdot $}
\renewcommand {\cftdotsep}{1.5}
\hypersetup{colorlinks=true,linkcolor=black}
\usepackage[a4paper, portrait, margin=2.5cm]{geometry}
\renewcommand{\baselinestretch}{1.25} %行间距取多倍行距(设置值为1.5)
\setlength{\baselineskip}{20pt} 

%% 页眉
\usepackage{fancyhdr}
\newcommand{\myname}{李佳}
\newcommand{\myid}{2100010793}
\pagestyle{fancy}
\fancyhf{}
\rhead{\myid}
\lhead{\myname}
\cfoot{\thepage}

%%%% Declare %%%
\DeclareMathOperator{\Ran}{Ran}
\DeclareMathOperator{\Dom}{Dom}
\DeclareMathOperator{\Rank}{Rank}

\newcommand{\md}{\mathrm{d}}
\newcommand{\mR}{\mathbb{R}}
\newcommand{\mbF}{\mathbb{F}}
%%% Declare %%%
\newtheorem{innercustomthm}{Problem}
\newenvironment{prob}[1]
{\renewcommand\theinnercustomthm{#1}\innercustomthm}
{\endinnercustomthm}

%%设置
\title{\textbf{数值分析$\ $上机作业3}}
\author{李佳~2100010793}
\date{}

%%正文
\begin{document}
\zihao{-4}
\maketitle
\begin{section}{问题描述}
    对Dennis-Schnabel书中Appendix B中的例2,3,5进行数值求解, 其中例子分别为:
    
    \noindent\textbf{(2)} Extended Powell Singular Function
    
    a) $n=4m,m\in\mathbb{N}_+$;

    b) for $i=1,2,...,n/4$:
    $$f_{4i-3}(x) = x_{4i-3}+10x_{4i-2}$$
    $$f_{4i-2}(x) = \sqrt{5}(x_{4i-1}-x_{4i})$$
    $$f_{4i-3}(x) = (x_{4i-2}-2x_{4i-1})^2$$
    $$f_{4i-3}(x) = \sqrt{10}(x_{4i-3}-x_{4i})^2$$

    c) 初值$x_0=(3,-1,0,1,...,3,-1,0,1)$

    d) 真解$x_{\ast} = (0,0,0,0,...,0,0,0,0)$

\

    \noindent\textbf{(3)} Trignometric Function

    a) $n\in\mathbb{N}_+$;

    b) for $i=1,2,...,n$:
    $$f_i(x) = n-\sum_{j=1}^n\bigg(\cos x_j + i(1-\cos x_i) -\sin x_i \bigg) = n + ni(1-\cos x_i)-n\sin x_i -\sum_{j=1}^n\cos x_j$$

    c) 初值$x_0=(1/n,1/n,...,1/n)$

\

    \noindent\textbf{(5)} Wood Function(优化问题转化为非线性方程组求解问题)

    a) $n=4$;

    b) $$f(x)=100(x_1^2-x_2)^2+(1-x_1)^2+90(x_3^2-x_4)^2+(1-x_3)^2 $$
         $$+ 10.1((1-x_2)^2+(1-x_4)^2)+19.8(1-x_2)(1-x_4)$$

    c) 初值$x_0=(-3,-1,-3,-1)$

    d) 极小值点$x_\ast = (1,1,1,1)$
\end{section}
\begin{section}{算法描述}
\begin{subsection}{Newton法}
    Newton迭代法使用泰勒展开线性近似($F(x)\approx F(x_0)+J_F(x_0)(x-x_0)$, $J_F(x_0)$为$x_0$处Jacobi矩阵)的方法, 求出一个近似零点, 再在该点附近线性近似得到下一个近似零点, 以此类推得到迭代序列. 
    具体来说, 迭代法为:
    $$x_{k+1} = x_k - J_F(x_k)^{-1}F(x_k).$$

    在$F(x)$二阶连续可微且零点$x_\ast$处的Jacobi矩阵非奇异时, Newton法有二阶局部收敛性; 

    该方法的缺点是每一步需要计算Jacobi矩阵, 并解一个线性方程组, 一般需要$O(n^3)$的更新代价($n$为方程个数), 因此不适合大规模计算.
\end{subsection}
\begin{subsection}{改进的Newton法: Broyden法}
    改进的Newton法:Broyden法希望降低每一步求$J_F(x_k)^{-1}$的代价, 认为在靠近零点时, 矩阵的变化很小, 希望通过秩一的更新来近似$J_F(x_k)$, 这样再通过Sherman-Morrison公式即可得到
    $J_F(x_k)^{-1}$.

    具体来说, 记第$k$步近似的Jacobi矩阵为$A_k$, 则希望寻找向量$v,w$使得$A_{k-1}+vw^T$近似为Jacobi矩阵, 
    即要求$F(x_k)-F(x_{k-1}) = (A_{k-1}+vw^T)(x_k-x_{k-1})$精确成立.

    这样每一步的更新代价是$O(n^2)$, 与原有的Newton法相比更具有实用性. Broyden法在$F$满足一定条件时可达到局部超线性收敛.

    另外, $v,w$的取法不唯一. 以下给出两种常用的取法, 也是本报告中涉及的取法:

    1. $w=Y_{k-1}:=x_k-x_{k-1}$, 此时可计算出$v=\frac{1}{\|Y_{k-1}\|_2^2}F(x_k)$(该取法的好处: $\|Y_{k-1}\|_2=0$等价于迭代停止), 通过Sherman-Morrison公式计算得到
    $$A_k^{-1} = A_{k-1}^{-1}-\frac{A_{k-1}^{-1}F(x_k)Y_{k-1}A_{k-1}^{-1}}{Y_{k-1}^TY_{k-1}+Y_{k-1}^TA_{k-1}^{-1}F(x_k)}$$

    2. $w=F(x_k)$, 此时有$v=\frac{1}{F(x_k)^TY_{k-1}}F(x_k)$(该取法的好处: $A_k$恒为对称矩阵, 适合近似优化问题的Hesse矩阵), 类似地可以算出
    $$A_k^{-1} = A_{k-1}^{-1}-\frac{A_{k-1}^{-1}F(x_k)F(x_k)^TA_{k-1}^{-1}}{F(x_k)^TY_{k-1}+F(x_k)A_{k-1}^{-1}F(x_k)}.$$
\end{subsection}
\begin{subsection}{为优化问题求解而改进的Newton法}
    由于优化问题中要求的是极小值, 而梯度为0的位置既有可能是极小值、也可能是极大值、也可能不是极值点, Hesse矩阵正定时能保证其是极小值, 因此直接对梯度为0的方程使用Newton法或Broyden方法可能无法收敛到极小值点. 
    在参考Dennis-Schnabel原书(Numerical Methods For Unconstrained Optimization and Nonlinear Equations)5.5节后, 发现可以对Newton法改进以使得迭代序列收敛于极小值点:

    令$f(x)$为待优化的函数, 原本的Newton法为:
    $$x_{k+1}=x_k-\nabla^2 f(x_k)^{-1}\cdot \nabla f(x_k),$$

    改进后为:
    $$x_{k+1}=x_k-(\nabla^2 f(x_k)+\mu_kI)^{-1}\cdot \nabla f(x_k),$$

    其中$\mu_k\geq 0$使得$\nabla^2 f(x_k)+\mu_kI$正定(实际计算中, 仅取$\mu_k$为一个正的常数).

    (由于本上机报告是非线性方程数值解的上机作业, 因此不再考虑其它与牛顿法联系不大的优化方面的算法.)
\end{subsection}
\end{section}

\begin{section}{计算结果}
    对例2(Powell Function), 计算$n=4$的情形(事实上, 该问题的$4m$个变量的方程组, 等价于相同的$m$组4个变量的方程组, 因此只以$n=4$为例求解). 由于已知真解, 因此迭代停止的标准设置为
    误差的2范数不大于$10^{-8}$. 采用Newton法和Broyden法(取法1)计算, 均收敛于真解$(0,0,0,0)$, 且误差2-范数随迭代步数的变化如图所示:
    \begin{figure}[!htbp]
        \centering
    \begin{tikzpicture}[]
        \begin{semilogyaxis}[ xlabel=迭代步数$k$,  ylabel=误差2-范数] % sharp plot: 折线图,通过修改此类型,即可完成多种图形绘制
            \addplot[mark=.,blue] coordinates{
                (1, 1.2263638342548486) (2, 0.6131819171274244) (3, 0.30659095856371227) (4, 0.1532954792818561) (5, 0.07664773964092805) (6, 0.03832386982046403) (7, 0.019161934910232013) (8, 0.009580967455116008) (9, 0.004790483727558005) (10, 0.002395241863779003) (11, 0.001197620931889502) (12, 0.000598810465944751) (13, 0.0002994052329723755) (14, 0.0001497026164861877) (15, 7.485130824309386e-05) (16, 3.742565412154693e-05) (17, 1.8712827060773464e-05) (18, 9.356413530386732e-06) (19, 4.678206765193366e-06) (20, 2.339103382596683e-06) (21, 1.1695516912983415e-06) (22, 5.847758456491708e-07) (23, 2.923879228245854e-07) (24, 1.461939614122927e-07) (25, 7.309698070614634e-08) (26, 3.654849035307317e-08) (27, 1.8274245176536586e-08) (28, 9.137122588268293e-09) 
                };\label{plot_one}
            \addplot[mark=.,red] coordinates{
                (1, 1.2263638342548486) (2, 0.8758484485582848) (3, 0.5109421361104696) (4, 0.32269318598791497) (5, 0.1977813816036325) (6, 0.12262405914416541) (7, 0.07569395756546181) (8, 0.04680311190858704) (9, 0.028920796081576647) (10, 0.01787524299106293) (11, 0.011047222537156794) (12, 0.006827626332880672) (13, 0.004219689243986408) (14, 0.0026079151272102525) (15, 0.001611779303379769) (16, 0.0009961346015987587) (17, 0.0006156449924650105) (18, 0.00038048954266029373) (19, 0.00023515546763515545) (20, 0.00014533407293998798) (21, 8.982139728956208e-05) (22, 5.551267710535957e-05) (23, 3.430872185181502e-05) (24, 2.120395678015854e-05) (25, 1.3104766484725164e-05) (26, 8.099191498045111e-06) (27, 5.005575854998989e-06) (28, 3.093615968398912e-06) (29, 1.9119593347151755e-06) (30, 1.1816546620364225e-06) (31, 7.303004035424312e-07) (32, 4.513462724554135e-07) (33, 2.789401306844987e-07) (34, 1.7238241032076875e-07) (35, 1.0651824638065184e-07) (36, 6.57992251291403e-08) (37, 4.061291581380978e-08) (38, 2.5013749256217882e-08) (39, 1.5319763998060336e-08) (40, 9.244928110209675e-09) 
                };\label{plot_two}
            \legend{Newton法, Broyden法}
        \end{semilogyaxis}
    \end{tikzpicture}
    \caption{例2使用Newton法、Broyden法时误差2-范数随迭代步数的变化图}
    \end{figure}

    对例3(Trignometric Function), 计算$n=10,20,30,40,50$的情形, 由于未知真解(事实上, 有无穷多真解, 且计算发现存在非平凡的解, 即不是$x_j=2k\pi$的解), 因此迭代停止的标准设置为
    相邻两步的步长2范数不大于$10^{-8}$, 即$\|x_k-x_{k-1}\|_2\leq 10^{-8}$.(当迭代法达到线性收敛时, $\|x_k-x_{k-1}\|_2$即可控制$\|x_k-x_\ast\|_2$的大小, 因此考虑用这个量作为停机标准).

    $n=10,20,30,40,50$时, 以初值为$(1/n,...,1/n)$, 分别采用Newton法和Broyden法(取法1)计算, 发现Newton法收敛, 但Broyden法不收敛. 在改变初值为$(1/n^3,...,1/n^3)$时, Broyden法可以收敛.
    当$n=10$时, 相邻两步的步长2-范数随迭代步数的变化如图所示; $n=10,20,30,40,50$时两方法达到停机标准的迭代次数如表所示.

    \begin{figure}[!htbp]
        \centering
    \begin{tikzpicture}[]
        \begin{semilogyaxis}[ xlabel=迭代步数$k$,  ylabel=相邻两步步长2-范数,legend style={at={(1.8,0.5)},anchor=east}] % sharp plot: 折线图,通过修改此类型,即可完成多种图形绘制
            \addplot[mark=.,blue] coordinates{
                (1, 8.843333875867064) (2, 3.535162145337083) (3, 0.7808073495728408) (4, 0.30083103034351505) (5, 0.12133851556574178) (6, 0.03253060448167888) (7, 0.0027808158197973336) (8, 2.0756585883252106e-05) (9, 1.1713101939553616e-09) 
                };\label{plot_3}
            \addplot[mark=.,red] coordinates{
                (1, 8.843333875867064) (2, 8.846230982639677) (3, 7.929228740516008) (4, 9.604574701042585) (5, 15.848773462292028) (6, 292.7359350460166) (7, 121.11166347820196) (8, 153.31932975991762) (9, 440.03685552293433) (10, 135.27183335377038) (11, 268.32743940392464) (12, 1433.5758248246127) (13, 2804.346241063161) (14, 1773.9490947631396) (15, 3091.1026775459945) (16, 4019.546131377103) (17, 12267.6506102374) (18, 129223.01967731574) (19, 590700.8982424393) (20, 1351087.4044278562) (21, 10103202.951956356) (22, 15035623.67605806) (23, 72476130.3326541) (24, 143668290.58541906) (25, 5694912621.987276) (26, 6567174243.964526) (27, 13089157620.238247) (28, 9289573276.19921) (29, 86352086002.0059) (30, 2531198937514.639) (31, 14217550142087.621) (32, 16157273183755.895) (33, 21761728111456.848) (34, 8764266443878.8) (35, 12655954897049.816) (36, 46685851965626.11) (37, 15897843948635.203) (38, 39091161037243.586) (39, 73181011309587.84) (40, 34291173876581.26) 
                };\label{plot_4}
            \addplot[mark=.,orange] coordinates{
                (1, 0.17491868705209196) (2, 0.17478835668483864) (3, 0.06710478065463728) (4, 0.058008047575764804) (5, 0.02404272295868699) (6, 0.019376912566452915) (7, 0.007499040766865112) (8, 0.005045424915132405) (9, 0.000990352314984514) (10, 0.00058007827419539) (11, 7.222463749695383e-05) (12, 3.4713318951365486e-05) (13, 4.16149061308905e-06) (14, 1.4115669171708765e-06) (15, 4.754242951843776e-07) (16, 2.7766310916287545e-07) (17, 9.143845851709606e-08) (18, 1.4571807881109228e-08) (19, 4.1678244206212673e-10) 
                };\label{plot_5}
            \legend{Newton法, Broyden法(初值($1/n$)),Broyden法(初值($1/n^3$))}
        \end{semilogyaxis}
    \end{tikzpicture}
    \caption{例3在$n=10$时, 各方法相邻两步的步长2-范数随迭代步数的变化}
    \end{figure}
        
    \begin{table}[!htbp]
        \centering
        \begin{tabular}{c|ccccc}
                                                                                & $n=10$ & $n=20$ & $n=30$ & $n=40$ & $n=50$ \\
                                                                                \hline
        \begin{tabular}[c]{@{}c@{}}Newton法\\ 初值($1/n,...,1/n$)\end{tabular}     & 9      & 10     & 11     & 10     & 11     \\
        \begin{tabular}[c]{@{}c@{}}Broyden法\\ 初值($1/n^3,...,1/n^3$)\end{tabular} & 19     & 24     & 30     & 36     & 43    \\
         \hline   
    \end{tabular}
    \caption{例3各方法达到停机标准的迭代次数}
    \end{table}

    例5(Wood Function)将优化问题转化为求解梯度为0的非线性方程组, 有可能不收敛于极小值点, 因此停机标准依然设置为相邻两步的步长2范数不大于$10^{-8}$, 即$\|x_k-x_{k-1}\|_2\leq 10^{-8}$. 
    
    计算发现Newton法、Broyden法(取法2)均收敛于一个鞍点$(-0.968, 0.947, -0.970, 0.951)$, Broyden法(取法1)可经过13702步的长时间迭代后收敛于极小值点$(1,1,1,1)$, 为优化问题改进的Newton法在取$\mu_k$为正的常数时大多可以收敛于极小值点.

    使用Newton法、Broyden法(取法2)、改进的Newton法(以$\mu_k=0.1$为例), 优化函数值$f(x_k)$随迭代步数$k$的变化如图所示; 使用Broyden法(取法1)时, 优化函数值$f(x_k)$随迭代步数$k$的变化如图所示; 
    改进的Newton法在$\mu_k$取不同的正的常数时达停机标准所需的迭代次数如表所示.

    \begin{figure}[!htbp]
        \centering
    \begin{tikzpicture}[]
        \begin{semilogyaxis}[ xlabel=迭代步数$k$,  ylabel=优化函数值$f(x_k)$,legend style={at={(1.8,0.5)},anchor=east}] % sharp plot: 折线图,通过修改此类型,即可完成多种图形绘制
            \addplot[mark=.,red] coordinates{
                (1, 1291.438570310242) (2, 295.9831774694114) (3, 67.80856254703502) (4, 17.377544970651208) (5, 8.697712430260678) (6, 7.8933616995772296) (7, 7.87655028154153) (8, 92.69245262504336) (9, 12.277170699070556) (10, 7.572884375262731) (11, 7.741155644324444) (12, 6.293193866647647) (13, 67650.95352944625) (14, 4902.006045819943) (15, 954.2809241601973) (16, 210.39431645423699) (17, 55.572195748218626) (18, 15.903036860648264) (19, 7.868335763457328) (20, 6.923896900249094) (21, 5.777873304712449) (22, 594.4325913765928) (23, 35.553121867673234) (24, 6.629133820693799) (25, 2.200474385826279) (26, 1.5855953854824598) (27, 6.859289086162104) (28, 0.558120632504572) (29, 35.078203489550106) (30, 0.691872196896111) (31, 0.08385755518894511) (32, 0.024208747124663976) (33, 0.0027463137656485725) (34, 0.00013131998924954943) (35, 2.2205521120786533e-06) (36, 3.2993652935112084e-08) (37, 4.908972882578581e-10) (38, 7.308299368379695e-12) (39, 1.0880433271499904e-13) (40, 1.6198548987492654e-15) (41, 2.411604251146965e-17) (42, 3.590343672939392e-19) 
                };\label{plot_6}
            \addplot[mark=.,blue] coordinates{
                (1, 1291.438570310242) (2, 295.95133777965924) (3, 67.6855947750042) (4, 17.336614172523827) (5, 8.689076695712231) (6, 7.892797749865293) (7, 7.87651605713956) (8, 7.877189874374443) (9, 7.876881954653394) (10, 7.876977155628779) (11, 7.876966626271154) (12, 7.8769671652519335) (13, 7.876967165176867) (14, 7.876967165176867) (15, 7.876967165176867)
                };\label{plot_7}
            \addplot[mark=.,green] coordinates{
                ( 1 , 1291.438570310242 )( 2 , 608.7617768114776 )( 3 , 190.35888608374455 )( 4 , 66.09786953302336 )( 5 , 46.132057103875205 )( 6 , 21.335681853163422 )( 7 , 494.13763066442453 )( 8 , 15.008082919047784 )( 9 , 12.086404070050671 )( 10 , 8.224735922894348 )( 11 , 7.902148249654897 )( 12 , 7.875649075006867 )( 13 , 7.939020274139651 )( 14 , 7.874838645236504 )( 15 , 7.8748380109502465 )( 16 , 7.874839178097841 )( 17 , 7.874099234522822 )( 18 , 7.87941353716523 )( 19 , 8.049503254157639 )( 20 , 7.876768501265439 )( 21 , 7.876732492605183 )( 22 , 7.876325529540618 )( 23 , 7.877039812584821 )( 24 , 7.8770178740386285 )( 25 , 7.877123849432559 )( 26 , 7.8769831546685785 )( 27 , 7.876952018890702 )( 28 , 7.876967090080429 )( 29 , 7.876967165099431 )( 30 , 7.87696716517861 )( 31 , 7.876967165176876 )( 32 , 7.8769671651768665 )( 33 , 7.876967165176867 )
                };\label{plot_8}
            \legend{改进Newton法($\mu_k=0.1$),Newton法,Broyden法(取法2)}
        \end{semilogyaxis}
    \end{tikzpicture}
    \caption{例5使用Newton法、Broyden法(取法2)、改进的Newton法($\mu_k=0.1$), $f(x_k)$随迭代步数$k$的变化}
    \end{figure}

    \begin{figure}[!htbp]
        \centering
    \begin{tikzpicture}[]
        \begin{semilogyaxis}[ xlabel=迭代步数$k$,  ylabel=优化函数值$f(x_k)$, xtick={0,5000,10000,14000}] % sharp plot: 折线图,通过修改此类型,即可完成多种图形绘制
            \addplot[mark=.,blue] coordinates{
                ( 2 , 918.9146438729339 )( 52 , 7.8749516133867346 )( 102 , 7.8743772336473095 )( 152 , 7.873013754933172 )( 202 , 7.875997671537793 )( 252 , 7.868795854417401 )( 302 , 7.872886195969094 )( 352 , 7.868099577883303 )( 402 , 7.861192922299783 )( 452 , 7.859939078609271 )( 502 , 7.878666325508241 )( 552 , 7.7706576287927955 )( 602 , 7.7430106001942445 )( 652 , 7.5185480244394896 )( 702 , 6.922615991325671 )( 752 , 6.431745777111042 )( 802 , 6.769473761273183 )( 852 , 7.176974041687149 )( 902 , 7.319505071996986 )( 952 , 7.300138053736106 )( 1002 , 12.488323696472673 )( 1052 , 6.888468166364852 )( 1102 , 6.250348415001685 )( 1152 , 6.214064503740186 )( 1202 , 6.218070439373349 )( 1252 , 6.287254920472941 )( 1302 , 6.387891242535943 )( 1352 , 6.326369083299653 )( 1402 , 6.283521741902208 )( 1452 , 6.3048680969972235 )( 1502 , 6.36624046665294 )( 1552 , 6.4603128156017835 )( 1602 , 6.447006894447796 )( 1652 , 6.432391100225837 )( 1702 , 6.588124214469115 )( 1752 , 6.606423479766979 )( 1802 , 6.603644372422567 )( 1852 , 6.655782614109338 )( 1902 , 6.760606463683196 )( 1952 , 6.860412586511323 )( 2002 , 7.138874870957849 )( 2052 , 6.6703276916375 )( 2102 , 426.8345651444805 )( 2152 , 1.9794184523132223 )( 2202 , 2.4160356531372678 )( 2252 , 5.204361769352352 )( 2302 , 1.4395985181898396 )( 2352 , 2.0547446379793612 )( 2402 , 7.806329154069257 )( 2452 , 2.6260943377958945 )( 2502 , 2.6601760226334417 )( 2552 , 2.573899150390231 )( 2602 , 4.663937330154637 )( 2652 , 2.4117320747448403 )( 2702 , 2.529542370198026 )( 2752 , 3.307985163968276 )( 2802 , 0.4892995212473501 )( 2852 , 0.42401037410166786 )( 2902 , 0.42474422874895446 )( 2952 , 0.33049926144052844 )( 3002 , 0.34430596343565867 )( 3052 , 0.13766541592926806 )( 3102 , 27.849044081365463 )( 3152 , 0.1260675829647835 )( 3202 , 0.2177313408972008 )( 3252 , 0.22149244253861067 )( 3302 , 0.24194674792619253 )( 3352 , 0.33419057835863697 )( 3402 , 0.282185013897819 )( 3452 , 0.16106948505975494 )( 3502 , 0.17423644786835402 )( 3552 , 0.5537846819232772 )( 3602 , 0.26618565237197167 )( 3652 , 0.18589110385169683 )( 3702 , 0.1527092350088557 )( 3752 , 0.16058731747108146 )( 3802 , 0.16269218651007655 )( 3852 , 0.11971464963258782 )( 3902 , 0.13589236606027022 )( 3952 , 0.10775745635922895 )( 4002 , 0.10237803193137962 )( 4052 , 7.967551441463527 )( 4102 , 0.1118733513607193 )( 4152 , 0.03680498655446984 )( 4202 , 0.046566719710440796 )( 4252 , 0.03414258523349356 )( 4302 , 0.07246942795473355 )( 4352 , 0.022600410217194056 )( 4402 , 0.12698179639522822 )( 4452 , 0.11374919813312223 )( 4502 , 0.08464142138960673 )( 4552 , 0.08944991332767449 )( 4602 , 0.06776340345450586 )( 4652 , 0.047812265914152885 )( 4702 , 0.03919768060830009 )( 4752 , 0.49157618839820993 )( 4802 , 0.25588314724992367 )( 4852 , 0.012788413003241406 )( 4902 , 0.016112228755908786 )( 4952 , 0.25744648706025974 )( 5002 , 0.04791691675838128 )( 5052 , 15.088891604002956 )( 5102 , 172.46622950518724 )( 5152 , 0.42130162196715126 )( 5202 , 0.4347510292112826 )( 5252 , 0.49610762687705723 )( 5302 , 0.3840627460043251 )( 5352 , 0.3261319655856614 )( 5402 , 28765.87295158289 )( 5452 , 0.3222062891291513 )( 5502 , 0.26884866019860265 )( 5552 , 0.33116745464228714 )( 5602 , 0.2778296069548194 )( 5652 , 0.27499264435119475 )( 5702 , 0.31365015507252014 )( 5752 , 0.2917371588474227 )( 5802 , 0.17830433625751407 )( 5852 , 0.2638265290671393 )( 5902 , 0.16370050575600414 )( 5952 , 0.3783548047939025 )( 6002 , 0.43307789562386034 )( 6052 , 0.3780646633544702 )( 6102 , 0.20035183919803057 )( 6152 , 0.19746956380978542 )( 6202 , 0.1890950000882623 )( 6252 , 0.20480000969107426 )( 6302 , 0.21738573703316177 )( 6352 , 0.21650676470296926 )( 6402 , 0.17650610677529732 )( 6452 , 0.1977904390080134 )( 6502 , 0.22643787880662014 )( 6552 , 0.14238531454928083 )( 6602 , 0.1746359239807287 )( 6652 , 0.08693001665315103 )( 6702 , 0.07269020872441678 )( 6752 , 0.1217310813033694 )( 6802 , 0.0817284801028566 )( 6852 , 20.816136684571305 )( 6902 , 8.21323309695495 )( 6952 , 0.5711986888374572 )( 7002 , 0.38945328221093867 )( 7052 , 0.26991010131610516 )( 7102 , 0.25297438714737375 )( 7152 , 0.24198590474883286 )( 7202 , 0.31481200527939546 )( 7252 , 0.26669576201726475 )( 7302 , 0.2348593610065688 )( 7352 , 0.30136024788123894 )( 7402 , 0.20629044436805266 )( 7452 , 0.33236324866071687 )( 7502 , 0.5055816688028827 )( 7552 , 0.06539350297102417 )( 7602 , 0.18929046663837923 )( 7652 , 0.18004241752324202 )( 7702 , 0.25420982203584686 )( 7752 , 0.21644779116401658 )( 7802 , 0.22240315061619054 )( 7852 , 0.22760821978022605 )( 7902 , 0.20037065526421838 )( 7952 , 0.23472638287509806 )( 8002 , 0.17265307535420638 )( 8052 , 0.29677207410560413 )( 8102 , 0.31017595779557894 )( 8152 , 0.24531195148747464 )( 8202 , 0.26867993443500016 )( 8252 , 0.3424872838898858 )( 8302 , 0.14414584288646282 )( 8352 , 0.23614947340866888 )( 8402 , 0.2669195401781961 )( 8452 , 0.23484302727040873 )( 8502 , 0.24814906270340842 )( 8552 , 0.2258345168544782 )( 8602 , 0.22174048194185758 )( 8652 , 0.22262591851046132 )( 8702 , 0.22354325812902776 )( 8752 , 0.22209700612578498 )( 8802 , 0.21003263488762247 )( 8852 , 0.2158227992671895 )( 8902 , 0.2672489355196266 )( 8952 , 0.2683990196861039 )( 9002 , 0.2529671000586289 )( 9052 , 0.22131254391784427 )( 9102 , 0.21692197020342086 )( 9152 , 0.21584807377030302 )( 9202 , 0.22130094302033676 )( 9252 , 0.11916808695474046 )( 9302 , 0.1623547561792602 )( 9352 , 0.20057246468211964 )( 9402 , 0.21370944827113547 )( 9452 , 0.18706782826391777 )( 9502 , 0.20292310725216733 )( 9552 , 0.19729531807779477 )( 9602 , 0.19565068297575916 )( 9652 , 0.18432511506858384 )( 9702 , 0.1855093283641689 )( 9752 , 0.20625333317680816 )( 9802 , 0.20225832297838586 )( 9852 , 0.20891911009061204 )( 9902 , 0.2051498244096681 )( 9952 , 0.18000456934130193 )( 10002 , 0.19671005226806404 )( 10052 , 0.22726605919044474 )( 10102 , 0.22390244598497855 )( 10152 , 0.22947914426953986 )( 10202 , 0.22762361087487726 )( 10252 , 93.82069297638213 )( 10302 , 0.28443149039034754 )( 10352 , 0.39383786069427096 )( 10402 , 0.42776927190645253 )( 10452 , 0.6854344212307257 )( 10502 , 0.5739983202812198 )( 10552 , 0.423243466036336 )( 10602 , 0.6279823890917093 )( 10652 , 234.40488992074262 )( 10702 , 0.5400600311907748 )( 10752 , 0.4820733494156073 )( 10802 , 0.5458913057100805 )( 10852 , 0.4262951576943088 )( 10902 , 0.4116668710596727 )( 10952 , 0.37293776002159884 )( 11002 , 0.3779097898717172 )( 11052 , 0.30556019079403907 )( 11102 , 0.3257701095033054 )( 11152 , 0.27341289397209767 )( 11202 , 0.2826202198440848 )( 11252 , 0.32147117067968445 )( 11302 , 0.2571891074688386 )( 11352 , 0.21743702854646596 )( 11402 , 0.20102362883038705 )( 11452 , 0.25368613923942274 )( 11502 , 0.16848703773872353 )( 11552 , 0.25014125796305375 )( 11602 , 0.2611388638757557 )( 11652 , 0.7127306020282678 )( 11702 , 0.2949813398800334 )( 11752 , 0.11569057855922349 )( 11802 , 0.22943596036225866 )( 11852 , 0.24417802083117746 )( 11902 , 0.15667398256273302 )( 11952 , 0.2871914689026518 )( 12002 , 0.3098103991446992 )( 12052 , 0.28970525183050366 )( 12102 , 0.2964398439438334 )( 12152 , 0.3076219895228043 )( 12202 , 0.35152332786918716 )( 12252 , 2.4387296672036456 )( 12302 , 0.2830693311024044 )( 12352 , 0.1568540046514144 )( 12402 , 0.2641102787499845 )( 12452 , 1.3011174146594175 )( 12502 , 0.1986901986745364 )( 12552 , 0.2210656510552047 )( 12602 , 0.23265167343399273 )( 12652 , 0.16513705555740898 )( 12702 , 0.0795742970381339 )( 12752 , 160.3904052864347 )( 12802 , 0.5206037869644842 )( 12852 , 0.47925828368724943 )( 12902 , 0.031244144742168856 )( 12952 , 0.8809548160298161 )( 13002 , 0.3366376468738901 )( 13052 , 2.0128468973284734 )( 13102 , 0.007498132420387826 )( 13152 , 0.0008459474982465305 )( 13202 , 0.023017522309831284 )( 13252 , 0.06228886082838114 )( 13302 , 0.1611555835908054 )( 13352 , 0.004867672652148547 )( 13402 , 120078.12315930412 )( 13452 , 2.8808809347053087e-05 )( 13502 , 0.0004032124646199606 )( 13552 , 0.03509708054115779 )( 13602 , 0.10755973053199996 )( 13652 , 0.003262614161583244 )( 13702 , 4.3184703478750064e-19 )
                };\label{plot_8}
            %\legend{Broyden法}
        \end{semilogyaxis}
    \end{tikzpicture}
    \caption{例5使用Broyden法(取法1), $f(x_k)$随迭代步数$k$的变化}
    \end{figure}

    \begin{table}[!htbp]
        \centering
        \begin{tabular}{ccccccccccc}\hline
        $\mu_k$ & 0.01 & 0.02 & 0.05 & 0.06 & 0.07 & 0.08 & 0.09 & 0.10 & 0.20 & 0.30 \\
        迭代次数    & 45   & 41   & 41   & 41   & 35   & 58   & 38   & 42   & 44   & 45   \\
        \hline
        $\mu_k$ & 0.40 & 0.50 & 0.60 & 0.70 & 0.80 & 0.90 & 1.00 & 2.00 & 3.00 & 4.00 \\
        迭代次数    & 55   & 53   & 56   & 63   & 68   & 70   & 75   & 120  & 162  & 204 \\
        \hline
        \end{tabular}
        \caption{例5中改进的Newton法在$\mu_k$取不同的正的常数时达停机标准所需的迭代次数}
        \end{table}
\end{section}
\newpage
\begin{section}{简明分析}
    \noindent 对例2(Powell Function), 由图可看出:
    \begin{itemize}
        \item 由于零点处Jacobi矩阵奇异, 因此Newton法无法达到二阶收敛, 但Newton法和Broyden法均能达到线性收敛; 
        \item Newton法的收敛速度快于Broyden法, 所需迭代次数更少.
    \end{itemize}

    \noindent 对例3(Trignometric Function), 由图表可看出:
    \begin{itemize}
        \item 可能受附近多个零点的影响, 初值为$(1/n,...,1/n)$时, Broyden法近似的Jacobi矩阵的误差让$x_k$不断向外, 继而导致Jacobi矩阵的近似变差, 从而又导致$x_k$不断偏离且步长越来越大, 导致无法收敛;
        \item 初值为$(1/n^3,...,1/n^3)$时, 离一个零点$(0,...,0)$的距离更近, 这使得Broyden法更容易收敛;
        \item 初值为$(1/n,...,1/n)$时, Newton法均能收敛. 与前两条分析对比可知, 该例子中Newton法对初值的要求低于Broyden法.
        \item 在迭代法收敛时, 它们均至少线性收敛; 
        \item 在$n$变化时, Newton法达到停机要求所需的迭代次数几乎不变, Broyden法(初值为$(1/n^3,...,1/n^3)$)所需迭代次数近似线性增长.
    \end{itemize}

    \noindent 对例5(Wood Function), 由图表可看出:
    \begin{itemize}
        \item Newton法、Broyden法(取法2)均没能得到极小值点, 均收敛于鞍点, 这是用非线性方程组求解优化问题所不可克服的问题, 因为梯度为0不是极小值点的充分条件, 
        要确保得到极小值点, 必须利用Hesse矩阵的信息并利用优化方面的知识;
        \item Broyden法(取法1)虽然收敛于极小值点, 但由图可以看出其大部分时间在鞍点处徘徊, 到最后才快速到达极小值点, 猜测这可能是由于迭代过程有一定的不稳定性, 异常的波动恰好将点拉到了极小值点附近;
        \item 改进的Newton法在$\mu_k$取为正的常数时, 大多数情况都能收敛于极小值点. 这可能是因为在鞍点处, $\mu_k$项会产生扰动, 使点远离鞍点, 促使点向极小值点移动;
        \item 在$\mu_k\leq 0.1$时, 迭代次数大多在40次上下, $\mu_k>0.10$后, 迭代次数随$k$呈上升趋势, 且$\mu_k>1.00$后近似为线性增长. 这可能是因为$\mu_k$较大时, 使每一步的步长减少, 导致迭代次数增多.
    \end{itemize}
\end{section}
\end{document}